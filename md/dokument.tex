\documentclass{vorlage-design-main}
\usepackage[utf8]{inputenc}
\usepackage{longtable}
\usepackage{blindtext,alltt}

\makeatletter
\newlength{\cslhangindent}
\setlength{\cslhangindent}{1.5em}
\newenvironment{CSLReferences}%
  {\setlength{\parindent}{0pt}%
   \everypar{\setlength{\hangindent}{\cslhangindent}}\ignorespaces}%
  {\par}
\makeatother


%% Ganze Überschrift
\title{Langer Titel}

%% Kürzerer Titel zur Verwendung im Seitenkopf
\runningtitle{Kurztitel}
\author{Jan Unger}
% \author{2.}
\date{\today}

%% Die .bib-Datei mit vollständigen Referenzen zur Verwendung mit biblatex. articleclass lädt das Paket biblatex-chicago mit Anpassungen
\addbibresource{literatur.bib}

\begin{document}

\maketitle

\begin{abstract}
Hier steht die Zusammenfassung des Dokuments. Sie kann über mehrere
Zeilen gehen und unterstützt auch Markdown-Formatierungen.
\end{abstract}

\section{Thema}\label{thema}

Text, der sich auf die Quelle bezieht {[}1{]}.

Hier ist ein Text, der eine Fußnote benötigt.\footnote{Hier steht der
  Text der Fußnote.}

\phantomsection\label{refs}
\begin{CSLReferences}{0}{0}
\bibitem[\citeproctext]{ref-spanner:2019:robotik}
\CSLLeftMargin{{[}1{]} }%
\CSLRightInline{G. Spanner, \emph{Robotik und künstliche intelligenz}.
2019.}

\end{CSLReferences}

%% Anhang
%\clearpage
%\appendix

\clearpage
\printbibliography
\end{document}
