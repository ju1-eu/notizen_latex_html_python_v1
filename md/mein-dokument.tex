\documentclass{vorlage-design-main}
\usepackage[utf8]{inputenc}
\usepackage{longtable}
\usepackage{blindtext,alltt}

%% Ganze Überschrift
\title{Langer Titel}

%% Kürzerer Titel zur Verwendung im Seitenkopf
\runningtitle{Kurztitel}
\author{Jan Unger}
% \author{2.}
\date{\today}

%% Die .bib-Datei mit vollständigen Referenzen zur Verwendung mit biblatex. articleclass lädt das Paket biblatex-chicago mit Anpassungen
\addbibresource{literatur.bib}

\providecommand{\tightlist}{%
  \setlength{\itemsep}{0pt}\setlength{\parskip}{0pt}}


\begin{document}

\maketitle

\begin{abstract}
Hier steht die Zusammenfassung des Dokuments. Sie kann über mehrere
Zeilen gehen und unterstützt auch Markdown-Formatierungen.
\end{abstract}

\section{Thema}\label{thema}

\begin{itemize}
\tightlist
\item
  Fachbuchautor \textcite{dalwigk:2024:fachbuchautor}.
\item
  Online Kurse \textcite{schaffranek:2024:kurse}.
\item
  Hacking und Cyber Security mit KI \textcite{dalwigk:2023:hacking}.
\item
  Python für Einsteiger \textcite{dalwigk:2022:python}.
\item
  Mikrocontroller ESP32 \textcite{brandes:2023:mikrocontroller}.
\item
  Roboterauto \textcite{brandes:2022:esp32}.
\item
  Daten mit Raspberry Pi im Netz speichern und visualisieren
  \textcite{brandes:2023:daten}.
\end{itemize}

Hier ist ein Text, der eine Fußnote benötigt.\footnote{Text der Fußnote.}

Test

\begin{enumerate}
\def\labelenumi{\arabic{enumi}.}
\tightlist
\item
  eins
\item
  zwei
\end{enumerate}

%% Anhang
%\clearpage
%\appendix

\clearpage
\printbibliography
\end{document}
