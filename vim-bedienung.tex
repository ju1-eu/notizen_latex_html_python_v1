% ju 05-Feb-24 vim-bedienung.tex
\documentclass{vorlage-design-main}
\usepackage[utf8]{inputenc}
\usepackage{longtable}
\usepackage{blindtext,alltt}
%% Ganze Überschrift
\title{Thema}

%% Kürzerer Titel zur Verwendung im Seitenkopf
\runningtitle{Kurztitel}
\author{Jan Unger}
% \author{2.}
\date{\today}

%% Die .bib-Datei mit vollständigen Referenzen zur Verwendung mit biblatex. articleclass lädt das Paket biblatex-chicago mit Anpassungen
\addbibresource{literatur.bib}

\begin{document}

\maketitle

\begin{abstract}

\end{abstract}

\hypertarget{vim-editor---bedienung}{%
\section{Vim-Editor - Bedienung}\label{vim-editor---bedienung}}

\hypertarget{grundlegende-befehle}{%
\subsection{Grundlegende Befehle:}\label{grundlegende-befehle}}

\begin{enumerate}
\def\labelenumi{\arabic{enumi}.}
\item
  \textbf{Starten und Beenden}:

\begin{lstlisting}[language=bash]
vi dateiname oder vim dateiname
# Beenden ohne Änderungen zu speichern - Drücke  Esc
:q!
# Speichern von Änderungen - Drücke Esc
:wq
\end{lstlisting}
\item
  \textbf{Modi}:

  \begin{itemize}
  
  \item
    \textbf{Normalmodus}: Der Standardmodus. Hier können Sie
    Navigations- und Bearbeitungsbefehle ausführen.
  \item
    \textbf{Einfügemodus}: Fügen Sie Text in die Datei ein.

    \begin{itemize}
    
    \item
      \verb|i| (um vor dem Cursor einzufügen)
    \item
      \verb|a| (um nach dem Cursor einzufügen)
    \item
      \verb|Esc|, um zum Normalmodus zurückzukehren.
    \end{itemize}
  \item
    \textbf{Befehlszeilenmodus}: Geben Sie Befehle wie Speichern oder
    Beenden ein.

    \begin{itemize}
    
    \item
      \verb|:| im Normalmodus, um in diesen Modus zu
      wechseln.
    \end{itemize}
  \end{itemize}
\item
  \textbf{Bewegung im Normalmodus}:

  \begin{itemize}
  
  \item
    Bewegen Sie den Cursor mit

    \begin{itemize}
    
    \item
      \verb|h| (links), \verb|j|
      (unten), \verb|k| (oben) und
      \verb|l| (rechts).
    \end{itemize}
  \item
    Springen Sie zum Anfang der Datei mit

    \begin{itemize}
    
    \item
      \verb|gg| und zum Ende mit
      \verb|G|
    \end{itemize}
  \item
    Springen Sie zum Anfang oder Ende einer Zeile mit

    \begin{itemize}
    
    \item
      \verb|0| bzw. \verb|$|
    \end{itemize}
  \end{itemize}
\item
  \textbf{Bearbeiten im Normalmodus}:

  \begin{itemize}
  
  \item
    Löschen Sie einen Zeichen mit

    \begin{itemize}
    
    \item
      \verb|x|
    \end{itemize}
  \item
    Löschen Sie eine Zeile mit

    \begin{itemize}
    
    \item
      \verb|dd|
    \end{itemize}
  \item
    Kopieren (yank) Sie eine Zeile mit

    \begin{itemize}
    
    \item
      \verb|yy|
    \end{itemize}
  \item
    Fügen Sie eine kopierte oder gelöschte Zeile ein

    \begin{itemize}
    
    \item
      \verb|p|
    \end{itemize}
  \end{itemize}
\item
  \textbf{Suchen und Ersetzen}:

  \begin{itemize}
  
  \item
    Suchen Sie nach einem Wort, indem Sie

    \begin{itemize}
    
    \item
      \verb|/wort| eingeben und
      \verb|Enter| drücken.
    \end{itemize}
  \item
    Ersetzen Sie ein Wort global in der Datei mit

    \begin{itemize}
    
    \item
      \verb|:\%s/alt/neu/g|
    \end{itemize}
  \end{itemize}
\end{enumerate}

\hypertarget{suchen}{%
\subsection{Suchen:}\label{suchen}}

\begin{itemize}

\item
  \textbf{Vorwärts nach einem Wort suchen}:
  \verb|/Wort|
\item
  \textbf{Rückwärts nach einem Wort suchen}:
  \verb|?Wort|
\item
  \textbf{Zum nächsten Treffer}: \verb|n|
\item
  \textbf{Zum vorherigen Treffer}: \verb|N|
\end{itemize}

\hypertarget{suchen-ersetzen}{%
\subsection{Suchen \& Ersetzen:}\label{suchen-ersetzen}}

\begin{itemize}

\item
  \textbf{Alle Vorkommen in der Datei ersetzen}:

  \begin{itemize}
  
  \item
    \verb|:\%s/altesWort/neuesWort/g|
  \end{itemize}
\item
  \textbf{Alle Vorkommen zwischen bestimmten Zeilen ersetzen}:

  \begin{itemize}
  
  \item
    \verb|:StartZeile,EndZeile s/altesWort/neuesWort/g|
  \item
    (z.B. \verb|:5,10s/altes/neues/g|)
  \end{itemize}
\item
  \textbf{Ersetzen mit Bestätigung}:

  \begin{itemize}
  
  \item
    \verb|:\%s/altesWort/neuesWort/gc|
  \end{itemize}
\item
  \textbf{Erstes Vorkommen in jeder Zeile ersetzen}:

  \begin{itemize}
  
  \item
    \verb|:\%s/altesWort/neuesWort/|
  \end{itemize}
\item
  \textbf{Ersetzen in einem ausgewählten Bereich}:

  \begin{enumerate}
  \def\labelenumi{\arabic{enumi}.}
  
  \item
    Markieren Sie den Bereich im visuellen Modus

    \begin{itemize}
    
    \item
      (drücken Sie \verb|v| und bewegen Sie den
      Cursor).
    \end{itemize}
  \item
    Dann \verb|:s/altesWort/neuesWort/g|
  \end{enumerate}
\item
  \textbf{Vom Anfang der Datei bis zur aktuellen Position}:

  \begin{itemize}
  
  \item
    \verb|:1,.s/altesWort/neuesWort/g|
  \end{itemize}
\item
  \textbf{Von der aktuellen Position bis zum Ende}:

  \begin{itemize}
  
  \item
    \verb|:.,$s/altesWort/neuesWort/g|
  \end{itemize}
\end{itemize}

\hypertarget{navigieren}{%
\subsection{Navigieren:}\label{navigieren}}

\begin{itemize}

\item
  \textbf{Zum Anfang der Datei}: \verb|gg|
\item
  \textbf{Zum Ende der Datei}: \verb|G|
\end{itemize}

\hypertarget{inhalt-kopieren}{%
\subsection{Inhalt kopieren:}\label{inhalt-kopieren}}

\begin{itemize}

\item
  \textbf{Kopiere den gesamten Inhalt ohne Zeilennummern in die
  Zwischenablage}:

  \begin{itemize}
  
  \item
    \verb|:\%y+|
  \end{itemize}
\end{itemize} % Platzhalter

%% Optional Anhang
%\clearpage
%\appendix

\clearpage
\printbibliography
\end{document}