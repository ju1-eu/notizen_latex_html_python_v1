% ju 28-Nov-24 KI-Prompts.tex
\documentclass{vorlage-design-main}
% Verwenden von fontspec und unicode-math für XeLaTeX oder LuaLaTeX

%% Ganze Überschrift
\title{Methodik zur Textanalyse und Wissensverarbeitung}
%% Kürzerer Titel
\runningtitle{Textanalyse und Wissensverarbeitung}
\author{Jan Unger}
\date{\today}

%% Referenzen
\addbibresource{literatur.bib}

\begin{document}

\maketitle

\begin{abstract}
Die vorliegende Methodensammlung präsentiert einen strukturierten Ansatz
zur Analyse und Verarbeitung von Fachtexten. Der Prozess gliedert sich
in mehrere Kernbereiche: Die initiale Textanalyse nutzt Markdown und
\LaTeX{} für eine präzise Aufbereitung. Verschiedene
Visualisierungstechniken (ASCII, Mermaid, SVG) unterstützen das
Verständnis komplexer Zusammenhänge. Eine mehrstufige Qualitätssicherung
gewährleistet die technische und inhaltliche Korrektheit. Innovative
Lernmethoden wie Concept Mapping und Spaced Repetition fördern die
nachhaltige Wissensaneignung. Die Vertiefungsphase umfasst die
Erstellung von Fragenkatalogen und Glossaren sowie eine kritische
Reflexion. Diese Methodik ermöglicht eine effiziente und nachhaltige
Wissensverarbeitung.

\textbf{Keywords:} Fachtextanalyse, Visualisierungstechniken,
Lernmethoden, Wissensvertiefung
\end{abstract}

\section{Methodik zur Textanalyse und
Wissensverarbeitung}\label{methodik-zur-textanalyse-und-wissensverarbeitung}

bearbeitet am 2024-11-28

KI-Tools\footnote{KI-Assistenten: \href{https://claude.ai/new}{Claude}
  und \href{https://www.perplexity.ai/}{Perplexity}},
Krafthand\footnote{Krafthand: \href{https://www.krafthand.de/}{Homepage}
  und \href{https://www.krafthand.de/service/ausgaben/}{Ausgaben}},
PDF-Tool\footnote{\href{https://www.ilovepdf.com/de/entsperren_pdf}{iLovePDF}
  - PDF Entsperren} und Valeo\footnote{\href{https://www.valeo-techacademy.com/de/learn}{Valeo
  Tech Academy} - E-Learning für Kfz-Technik}.

\subsection{ToDo-Liste}\label{todo-liste}

\begin{itemize}

\item[$\boxtimes$]
  Krafthandausgabe 22

  \begin{itemize}

  \item[$\boxtimes$]
    HUK-E-Barometer zur E-Mobilität in Deutschland
  \item[$\boxtimes$]
    Drive Pilot von Mercedes-Benz - Erweiterung auf 95 km/h
  \item[$\boxtimes$]
    Spot-Repair in der Karosseriereparatur
  \item[$\boxtimes$]
    VW Golf 7: Problemanalyse Tankklappe
  \item[$\boxtimes$]
    Cell Caps und Cell Envelopes - Innovationen für HV-Batterien
  \end{itemize}
\item[$\square$]
  Krafthandausgabe 21

  \begin{itemize}

  \item[$\square$]
    ADAC-Winterreifentest 2024: Sieger und Verlierer unter den
    Winterreifen
  \item[$\square$]
    HV-Plattform PPE von Audi: Audis neue Plattform für E-Autos
  \item[$\square$]
    Check von GTÜ, ACE und ARBÖ: Ganzjahresreifen im Härtetest
  \item[$\square$]
    Gefahr schon bei 2,8 Grad: Wie exakt müssen Fahrerassistenzsysteme
    kalibriert werden?
  \item[$\square$]
    Interview Rowe zum Automatikgetriebeölwechsel: Expertenmeinung zum
    Ölverschleiß im Automatikgetriebe
  \item[$\square$]
    Wettbewerbswidrige OBD-Sperren: Das Dilemma mit den Security
    Gateways
  \end{itemize}
\item[$\square$]
  Krafthandausgabe 20

  \begin{itemize}

  \item[$\boxtimes$]
    LED-Nachrüstung von Halogenlampen
  \item[$\square$]
    Audi, Seat, Škoda und VW, diverse Modelle: Quietsch- und
    Knarzgeräusche bei vollem Lenkeinschlag
  \end{itemize}
\item[$\square$]
  Krafthandausgabe 18-19

  \begin{itemize}

  \item[$\square$]
    Batterietechnologie: Zellarten bei Cell-to-Pack
  \item[$\square$]
    Batterietechnologie: Was Cell-to-Pack bedeutet
  \item[$\square$]
    Umfrage bei 274 Kfz-Betrieben: Das Angebot der Werkstätten muss sich
    ändern
  \end{itemize}
\item[$\square$]
  Krafthandausgabe 17

  \begin{itemize}

  \item[$\square$]
    Statische Batterieprüfung: E-Autobatterie in 15 Minuten testen
  \item[$\square$]
    Hygienerichtlinie für Fahrzeugklimaanlagen: Mangelnde Luftqualität
    -- eine unsichtbare Gefahr
  \item[$\square$]
    Wartungsvorgaben nicht erfüllt: Garantieverweigerung (un-)möglich
    durch OEM
  \item[$\square$]
    Nachgefragt: Servicebedarf und Mischbarkeit von RDKS-Sensoren
  \end{itemize}
\end{itemize}

\newpage

\subsection{Exzerpieren: Textanalyse und
-zusammenfassung}\label{exzerpieren-textanalyse-und--zusammenfassung}

\begin{lstlisting}
Exzerpiere und beachte (exzerpieren-anweisung.md)
Thema:
Eingabe: (.pdf) im Anhang
Ausgabe: in Markdown und deutsch und Latex-Mathe

Erkläre [ ]

# Mindmap
Erstelle eine Mindmap in ASCII-Textform
Erstelle eine Mindmap als Mermaid-Diagramm
Erstelle eine Mindmap mit hellem Hintergrund, wie das Referenzbild
Erstelle eine SVG-Darstellung der Mindmap, verwende ein hierarchisches Layout mit verschiedenen Farben und Formen
Erstelle ein Flussdiagramm mit hellem Hintergrund, wie das Referenzbild

# Rechtschreibung
Prüfe Grammatik, Rechtschreibung, Klarheit, Prägnanz, Redundanz
Eingabe: [ ]

# Latex
Erstelle Latex ohne Präambel

# Code
Prüfe den Code
Erstelle docstring, deutsche Kommentare
Ertselle ein Code-Beschreibung
Erkläre den Code
Verbessere den Code mit Fokus auf:
  - Wartbarkeit, Redundanzverminderung, bessere Konsolenausgabe und Benutzerfreundlichkeit
\end{lstlisting}

\newpage

\subsection{Erarbeitung und Vertiefung von
Fachtexten}\label{erarbeitung-und-vertiefung-von-fachtexten}

\begin{enumerate}
\def\labelenumi{\arabic{enumi}.}
\item
  Erstellung einer Textanalyse
\item
  Erstellung eines Exzerpts
\item
  Erkläre
\item
  Erstellung eines Fragenkatalogs für Anki
\item
  Erstellung einer Bildanalyse
\item
  Erstellung zweier Mindmaps (strukturierte Liste)

  \begin{itemize}

  \item
    Eine für Zahlen, Daten und Fakten
  \item
    Eine für Fach- und Schlüsselwörter mit detaillierten Erklärungen
  \end{itemize}
\item
  Kritische Reflexion
\item
  Ergänzung um Fachbegriff-Glossar

  \begin{itemize}

  \item
    Strukturierte Lernziele
  \item
    Umfassendes Fachbegriff-Glossar
  \item
    Kapitelzusammenfassungen
  \end{itemize}
\item
  Erstelle mir die drei bis fünf wichtigsten Konzepte mit Erklärung
\item
  Lernmethoden

  \begin{itemize}

  \item
    CONCEPT MAPPING UND VERNETZTES DENKEN
  \item
    HISTORISCHE ENTWICKLUNG
  \item
    EXPERIMENTELLES LERNEN
  \item
    SYSTEMATISCHE WIEDERHOLUNG

    \begin{itemize}

    \item
      Prinzip >>spaced repetition<< Konzepte in größeren Zeitabständen
      wiederholen
    \end{itemize}
  \item
    Peer Learning: Konzepte erklären
  \end{itemize}
\end{enumerate}

%% Anhang
%\clearpage
%\appendix

\clearpage
\printbibliography
\end{document}
