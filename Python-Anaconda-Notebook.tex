% ju 05-Feb-24 Python-Anaconda-Notebook.tex
\documentclass{vorlage-design-main}
\usepackage[utf8]{inputenc}
\usepackage{longtable}
\usepackage{blindtext,alltt}
%% Ganze Überschrift
\title{Thema}

%% Kürzerer Titel zur Verwendung im Seitenkopf
\runningtitle{Kurztitel}
\author{Jan Unger}
% \author{2.}
\date{\today}

%% Die .bib-Datei mit vollständigen Referenzen zur Verwendung mit biblatex. articleclass lädt das Paket biblatex-chicago mit Anpassungen
\addbibresource{literatur.bib}

\begin{document}

\maketitle

\begin{abstract}

\end{abstract}

\hypertarget{python-und-anaconda-und-jupyter-notebook}{%
\section{Python und Anaconda und Jupyter
Notebook}\label{python-und-anaconda-und-jupyter-notebook}}

\hypertarget{jupyter-notebook}{%
\subsection{Jupyter Notebook}\label{jupyter-notebook}}

Jupyter Notebook ist eine interaktive Umgebung, die das Ausführen von
Python-Code in einer browserbasierten Anwendung ermöglicht.

\hypertarget{jupyter-notebook-starten}{%
\subsubsection{Jupyter Notebook
Starten}\label{jupyter-notebook-starten}}

\begin{lstlisting}[language=bash]
# Terminal
conda update --all
# Umgebungen auflisten
conda env list
conda activate base
conda deactivate
# Navigieren zum Projekt-Verzeichnis
jupyter notebook
\end{lstlisting}

\hypertarget{grundlegende-bedienung}{%
\subsubsection{Grundlegende Bedienung}\label{grundlegende-bedienung}}

\begin{itemize}

\item
  \textbf{Zellen}: Jupyter Notebooks bestehen aus Zellen, die entweder
  Code oder Markdown enthalten können.
\item
  \textbf{Ausführung von Zellen}:

  \begin{itemize}
  
  \item
    \verb|Shift + Enter|
  \end{itemize}
\item
  \textbf{Hinzufügen neuer Zellen}:

  \begin{itemize}
  
  \item
    \verb|Insert| \textgreater{}
    \verb|Insert Cell Above| oder
    \verb|Insert Cell Below|
  \end{itemize}
\end{itemize}

\hypertarget{python-code-in-einer-zelle}{%
\subsubsection{Python-Code in einer
Zelle}\label{python-code-in-einer-zelle}}

\begin{lstlisting}[language=Python]
# Code
print("Hallo Jupyter!")
# Einfache Rechenoperation
2 + 2
\end{lstlisting}

\hypertarget{markdown-fuxfcr-text-und-dokumentation}{%
\subsubsection{Markdown für Text und
Dokumentation}\label{markdown-fuer-text-und-dokumentation}}

\begin{lstlisting}
# Markdown
# Überschrift 1
### Überschrift 2
- oder * für Listen
[Linktext](URL)`
![Alt-Text](Bild-URL)
$2 + 2$
\end{lstlisting}

\hypertarget{magische-befehle}{%
\subsubsection{Magische Befehle}\label{magische-befehle}}

\textbf{Magische Befehle}

\begin{itemize}

\item
  \verb|\%time|: Zeigt die Ausführungszeit einer
  Zeile.
\item
  \verb|\%matplotlib inline|: Erlaubt das Anzeigen
  von Matplotlib-Diagrammen direkt im Notebook.
\end{itemize}

\hypertarget{interaktive-widgets}{%
\subsubsection{Interaktive Widgets}\label{interaktive-widgets}}

dynamische, interaktive Benutzeroberflächen erstellen.

\begin{lstlisting}[language=Python]
from ipywidgets import interact
def f(x):
    return x
interact(f, x=10)
\end{lstlisting}

\hypertarget{tastenkombinationen}{%
\subsubsection{Tastenkombinationen}\label{tastenkombinationen}}

\begin{itemize}

\item
  \verb|Shift + Enter|: Führe die aktuelle Zelle aus
  und gehe zur nächsten.
\item
  \verb|Esc|: Wechsle in den Kommandomodus.
\item
  \verb|M|: Ändere die Zelle in Markdown.
\item
  \verb|Y|: Ändere die Zelle in Code.
\end{itemize}

\hypertarget{anaconda}{%
\subsection{Anaconda}\label{anaconda}}

\textbf{Anaconda installieren}
\url{https://www.anaconda.com/products/individual}

\begin{lstlisting}[language=bash]
# Anaconda Navigator starten
anaconda-navigator
# Überprüfen der Anaconda-Installation
conda info
# Eine neue Conda-Umgebung erstellen
conda create --name <umgebungsname> python=<version>
# Aktivieren einer Conda-Umgebung
conda activate <umgebungsname>
# Deaktivieren der aktuellen Conda-Umgebung
conda deactivate
# Liste der installierten Pakete in der aktuellen Umgebung anzeigen
conda list
# Ein spezifisches Paket in der aktuellen Umgebung installieren
conda install <paketname>
# Ein spezifisches Paket in einer spezifischen Umgebung installieren
conda install --name <umgebungsname> <paketname>
# Anaconda-Umgebungen auflisten
conda env list
# Eine spezifische Anaconda-Umgebung entfernen
conda env remove --name <umgebungsname>
# Aktualisieren von Anaconda
conda update --all
# Anaconda-Pakete aktualisieren
conda update <paketname>
\end{lstlisting}

\hypertarget{workflow---jupyter-notebook}{%
\subsubsection{Workflow - Jupyter
Notebook}\label{workflow---jupyter-notebook}}

\begin{lstlisting}[language=bash]
# Anaconda und alle Pakete aktualisieren
conda update --all
# Umgebungen auflisten
conda env list
# Eine Umgebung entfernen
conda env remove --name meinenv
# Erstellen einer neuen Umgebung
conda create --name meinenv python=3.11
# Aktivieren der Umgebung
conda activate meinenv
# Installation von Paketen
conda install numpy pandas matplotlib
# Start von Jupyter Notebook
jupyter notebook
# Deaktivieren der Umgebung
conda deactivate
\end{lstlisting}

\hypertarget{workflow---python-script-in-einer-anaconda-umgebung}{%
\subsubsection{Workflow - Python-Script in einer
Anaconda-Umgebung}\label{workflow---python-script-in-einer-anaconda-umgebung}}

\begin{lstlisting}[language=bash]
# Erstelle eine neue Anaconda-Umgebung (optional, aber empfohlen):
conda create --name PythonGrundlagen_env python=3.11
# Anaconda-Umgebung aktivieren:
conda activate PythonGrundlagen_env
# Suche nach einem spezifischen Paket
conda list | grep PyQt5
# Installiere die benötigte Software
conda install pyqt
# Skript ausführen
python3 kfz_datenbank.py
# Deaktivieren einer Anaconda-Umgebung
conda deactivate
\end{lstlisting}

\textbf{Test grafische Benutzeroberfläche (GUI)}

\begin{lstlisting}[language=Python]
# name_script.py
# Test GUI
import sys
from PyQt5.QtWidgets import QApplication, QWidget

app = QApplication(sys.argv)
window = QWidget()
window.setWindowTitle('Testfenster')
window.show()
sys.exit(app.exec_())

# Terminal $ python3 name_script.py
\end{lstlisting}

\hypertarget{pep-8-stilrichtlinien}{%
\subsection{PEP 8-Stilrichtlinien}\label{pep-8-stilrichtlinien}}

\begin{enumerate}
\def\labelenumi{\arabic{enumi}.}
\item
  \textbf{Einrückung}: Verwenden Sie 4 Leerzeichen pro Einrückungsebene.
\item
  \textbf{Zeilenlänge}: Beschränken Sie alle Zeilen auf maximal 79 - 120
  Zeichen. Längere Zeilen sollten für verbesserte Lesbarkeit umgebrochen
  werden.
\item
  \textbf{Importe}:

  \begin{itemize}
  
  \item
    Importe sollten immer am Anfang einer Datei stehen.
  \item
    Reihenfolge:

    \begin{itemize}
    
    \item
      Standardbibliothek-Importe,
    \item
      Importe von Drittanbietern,
    \item
      lokale Anwendungs-/Bibliotheks-spezifische Importe.
    \end{itemize}
  \item
    Vermeiden Sie Wildcard-Importe

    \begin{itemize}
    
    \item
      \verb|from module import *|
    \end{itemize}
  \end{itemize}
\item
  \textbf{Leerzeichen in Ausdrücken und Anweisungen}:

  \begin{itemize}
  
  \item
    Unmittelbar vor einem Komma, Semikolon oder Doppelpunkt.
  \item
    Unmittelbar vor der Klammer, die eine Liste von Argumenten oder
    Index-Operatoren öffnet.
  \item
    Zwischen dem Funktionsnamen und der folgenden Klammer.
  \item
    Vor oder nach einem Zuweisungs- oder Vergleichsoperator.
  \end{itemize}
\item
  \textbf{Kommentare}: Kommentare sollten klar, präzise und so aktuell
  wie möglich gehalten werden. Kommentare sollten sich auf das Warum,
  nicht das Was konzentrieren.
\item
  \textbf{Benennungskonventionen}:

  \begin{itemize}
  
  \item
    Klassenname: \verb|CamelCase|
  \item
    Funktions- und Variablennamen: \verb|snake\_case|
  \item
    Konstanten: \verb|UPPER\_CASE|
  \end{itemize}
\item
  \textbf{Leerzeilen}:

  \begin{itemize}
  
  \item
    Verwenden Sie zwei Leerzeilen zwischen Funktionen und
    Klassendefinitionen.
  \item
    Verwenden Sie eine Leerzeile zwischen Methodendefinitionen innerhalb
    einer Klasse.
  \end{itemize}
\item
  \textbf{Leerzeichen um Operatoren}:

\begin{lstlisting}[language=Python]
= != < > :
# Nicht jedoch für Klammerungen und Indexierungen/Slices
() [] {}
list[index]
\end{lstlisting}
\item
  \textbf{Dokumentationsstrings (Docstrings)}:
\end{enumerate}

\begin{itemize}

\item
  Verwenden Sie dreifache doppelte Anführungszeichen für Docstrings.
\item
  Der erste Satz des Docstrings sollte kurz und eine zusammenfassende
  Beschreibung sein.
\end{itemize}

\begin{enumerate}
\def\labelenumi{\arabic{enumi}.}
\setcounter{enumi}{9}

\item
  \textbf{Dateistruktur und Organisation}:
\end{enumerate}

\begin{itemize}

\item
  Definieren Sie alle Imports am Anfang des Skripts.
\item
  Dann definieren Sie Konstanten.
\item
  Anschließend kommen Funktionen und Klassen.
\item
  Der ausführbare Teil des Skripts sollte ganz am Ende stehen

  \begin{itemize}
  
  \item
    \verb|if \_\_name\_\_ == "\_\_main\_\_":|
  \end{itemize}
\end{itemize}

\hypertarget{pruxfcfen-mit-tools-wie-flake8-oder-pylint}{%
\subsubsection{Prüfen mit Tools wie flake8 oder
pylint}\label{pruefen-mit-tools-wie-flake8-oder-pylint}}

\hypertarget{installation}{%
\paragraph{Installation}\label{installation}}

\begin{lstlisting}[language=bash]
pip install flake8
pip install pylint
\end{lstlisting}

\hypertarget{verwendung}{%
\paragraph{Verwendung}\label{verwendung}}

\begin{lstlisting}[language=bash]
flake8 script.py
pylint script.py
\end{lstlisting}

\hypertarget{python-und-anaconda-update}{%
\subsection{Python und Anaconda
Update}\label{python-und-anaconda-update}}

\textbf{Anaconda-Installation}

\begin{lstlisting}[language=bash]
# Anaconda
# Aktualisieren von Conda
conda info
conda list python
conda update conda
# Aktualisieren aller Pakete
conda update --all
# Installieren des Anaconda Metapakets (optional):
conda install anaconda
# aktualisieren
conda update anaconda
conda update jupyter
jupyter --version
conda update pandas matplotlib
conda info --envs
conda deactivate
conda create -n py312 python=3.12.1
conda activate base
conda update anaconda
conda activate py312
conda list python
# Python-Pakete installieren
conda install pillow
\end{lstlisting}

\textbf{Python-Installation}

\begin{lstlisting}[language=Python]
# Xcode Command Line Tools-Paket
xcode-select -p
#xcode-select --install
export CC=/usr/bin/clang
export CXX=/usr/bin/clang++

# Python-Version-Manager
# Python-Pfad
which python
which python3
# Überprüfe die Python-Version
python --version
python3 --version
pip3 --version
brew update
brew upgrade python
brew install pyenv
pyenv versions
#  pyenv initialisieren und Pfad zur Umgebungsvariablen hinzugefügen
# Homebrew-Link zu Python zu erneuern
#unalias python
#brew link --overwrite python
vim ~/.zshrc
    if command -v pyenv 1>/dev/null 2>&1; then
        eval "$(pyenv init -)"
    fi
    export PATH="/usr/local/bin:$PATH"
    export PATH="/Users/jan/.pyenv/versions/3.12.1/bin:$PATH"
    alias python=python3
source ~/.zshrc
# aktuelles Python 3.12.1 mit pyenv installieren
pyenv install 3.12.1
pyenv global 3.12.1
python --version
# Python-Pakete installieren
pip install Pillow
\end{lstlisting}

\hypertarget{homebrew-kurz-brew-dem-paketmanager-fuxfcr-macos}{%
\subsection{Homebrew (kurz >>brew<<), dem Paketmanager für
macOS}\label{homebrew-kurz-brew-dem-paketmanager-fuer-macos}}

\begin{lstlisting}[language=bash]
# Homebrew installieren
/bin/bash -c "$(curl -fsSL https://raw.githubusercontent.com/Homebrew/install/HEAD/install.sh)"
# Homebrew aktualisieren
brew update
brew install <paketname>
# Installierte Pakete auflisten
brew list
# Informationen über ein Paket anzeigen
brew info <paketname>
# Paket aktualisieren
brew upgrade <paketname>
brew uninstall <paketname>
# Überprüfen, ob Ihr System irgendwelche Probleme hat
brew doctor
# Suche nach Paketen
brew search <suchbegriff>
# Abhängigkeiten anzeigen
brew deps <paketname>
\end{lstlisting}
 % Platzhalter

%% Optional Anhang
%\clearpage
%\appendix

\clearpage
\printbibliography
\end{document}