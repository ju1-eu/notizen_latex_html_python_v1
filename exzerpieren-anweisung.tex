% ju 28-Nov-24 exzerpieren-anweisung.tex
\documentclass{vorlage-design-main}
% Verwenden von fontspec und unicode-math für XeLaTeX oder LuaLaTeX

%% Ganze Überschrift
\title{}
%% Kürzerer Titel
\runningtitle{}
\author{Jan Unger}
\date{\today}

%% Referenzen
\addbibresource{literatur.bib}

\begin{document}

\maketitle

\begin{abstract}



\end{abstract}

\section{Exzerpieren: Textanalyse und
-zusammenfassung}\label{exzerpieren-textanalyse-und--zusammenfassung}

\subsection{Definition und Herkunft}\label{definition-und-herkunft}

>>Exzerpt<< (lateinisch >>excerpere<<: auswählen) bezeichnet eine
wissenschaftliche Arbeitsmethode zur systematischen Erfassung und
Zusammenfassung von Texten. Diese Technik dient der präzisen Darstellung
von Kerninhalten und zentralen Argumenten eines Textes.

\subsection{Formen des Exzerpierens}\label{formen-des-exzerpierens}

\begin{enumerate}
\def\labelenumi{\arabic{enumi}.}

\item
  \textbf{Allgemeine Zusammenfassung}

  \begin{itemize}

  \item
    Systematische Erfassung der Kernthesen und Hauptargumente des
    gesamten Textes
  \item
    Überblicksdarstellung der wesentlichen Inhalte
  \end{itemize}
\item
  \textbf{Spezifische Analyse}

  \begin{itemize}

  \item
    Gezielte Filterung relevanter Inhalte für konkrete Forschungsfragen
  \item
    Fokussierte Auswertung spezifischer Textaspekte
  \end{itemize}
\end{enumerate}

\subsection{Bestandteile eines
Exzerpts}\label{bestandteile-eines-exzerpts}

\begin{itemize}

\item
  Vollständige bibliografische Angaben des Originaltexts
\item
  Strukturierte Zusammenfassung der Hauptinhalte
\item
  Dokumentation zentraler Argumente, Daten und Fakten
\item
  Klar gekennzeichnete eigene Anmerkungen und Querverweise
\end{itemize}

\subsection{Systematischer
Arbeitsprozess}\label{systematischer-arbeitsprozess}

\subsubsection{1. Vorbereitung}\label{vorbereitung}

\begin{itemize}

\item
  Erfassung der vollständigen Quellenangaben
\item
  Definition der Kernthemen und Forschungsfragen
\item
  Erstellung einer systematischen Dokumentstruktur
\end{itemize}

\subsubsection{2. Überblickslektüre}\label{ueberblickslektuere}

\begin{itemize}

\item
  Vollständiges Lesen des Originaltexts
\item
  Erfassung der Hauptthemen und Textstruktur
\item
  Dokumentation initialer Fragestellungen
\end{itemize}

\subsubsection{3. Detailanalyse}\label{detailanalyse}

\begin{itemize}

\item
  Systematische Textmarkierung
\item
  Identifikation von Kernaussagen und Schlüsselbegriffen
\item
  Hervorhebung zentraler Thesen
\item
  Markierung relevanter Belege und Zitate
\end{itemize}

\subsubsection{4. Strukturierung}\label{strukturierung}

\begin{itemize}

\item
  Logische Textgliederung
\item
  Analyse der Argumentationsstruktur
\item
  Identifikation von Hauptargumenten
\item
  Entwicklung einer klaren Gliederungsstruktur
\end{itemize}

\subsubsection{5. Zusammenfassung}\label{zusammenfassung}

\begin{itemize}

\item
  Präzise Wiedergabe relevanter Textpassagen
\item
  Einhaltung der logischen Argumentationsstruktur
\item
  Eindeutige Kennzeichnung und Belegung von Zitaten
\end{itemize}

\subsubsection{6. Kommentierung}\label{kommentierung}

\begin{itemize}

\item
  Integration eigener Gedanken und Analysen
\item
  Einfügung relevanter Querverweise
\item
  Dokumentation kritischer Anmerkungen
\item
  Klare Kennzeichnung eigener Überlegungen
\end{itemize}

\subsubsection{7. Qualitätskontrolle}\label{qualitaetskontrolle}

\begin{itemize}

\item
  Prüfung auf Vollständigkeit
\item
  Verifizierung aller Quellenangaben
\item
  Sicherstellung der Verständlichkeit
\item
  Überprüfung der Gesamtstruktur
\end{itemize}

\subsubsection{8. Finalisierung}\label{finalisierung}

\begin{itemize}

\item
  Erstellung eines Inhaltsverzeichnisses
\item
  Vergabe präziser Schlagworte
\item
  Digitale Archivierung in definierter Ordnerstruktur
\item
  Dokumentation von Datum und Version
\end{itemize}

\subsection{Beispiele und Textarbeit}\label{beispiele-und-textarbeit}

\subsubsection{Kernaussagen
identifizieren}\label{kernaussagen-identifizieren}

\textbf{Originaltext:} \emph{„Die globale Erwärmung ist vor allem auf
den erhöhten CO2-Ausstoß durch fossile Brennstoffe zurückzuführen. Dies
hat schwerwiegende Folgen für Ökosysteme und die menschliche
Gesellschaft.<<}

\textbf{Kernaussage:} \emph{„Hauptursache der globalen Erwärmung:
erhöhter CO2-Ausstoß durch fossile Brennstoffe; Konsequenzen:
Beeinträchtigung von Ökosystemen und Gesellschaft.<<}

\subsubsection{Eigene Anmerkungen
integrieren}\label{eigene-anmerkungen-integrieren}

\textbf{Originaltext:} \emph{„Die Verlagerung auf erneuerbare Energien
könnte diese Entwicklung verlangsamen.<<}

\textbf{Kommentierte Fassung:} \emph{„Die Autoren betonen das Potenzial
erneuerbarer Energien, berücksichtigen jedoch nicht die
sozioökonomischen Implementierungshürden wie Investitionskosten und
Infrastrukturanpassungen.<<}

\subsection{Qualitätssicherung}\label{qualitaetssicherung}

\subsubsection{Allgemeine
Qualitätskriterien}\label{allgemeine-qualitaetskriterien}

\begin{itemize}

\item[$\square$]
  Vollständige bibliografische Dokumentation
\item[$\square$]
  Präzise Seitenangaben bei Zitaten
\item[$\square$]
  Systematische Dokumentenstruktur
\item[$\square$]
  Klare Trennung von Original und Eigenleistung
\item[$\square$]
  Eindeutige Kennzeichnung von Zitaten
\end{itemize}

\subsubsection{Fachspezifische
Dokumentation}\label{fachspezifische-dokumentation}

\paragraph{Naturwissenschaften
(Physik/Mathematik)}\label{naturwissenschaften-physikmathematik}

\textbf{Dokumentationsanforderungen:} - Mathematische Formeln und
Herleitungen - Experimentelle Methoden und Aufbauten - Messwerte mit
Fehlerrechnung - Einheitenkonsistenz - Theoretische Grundlagen

\textbf{Qualitätskriterien:} - {[} {]} Mathematische Notation
einheitlich - {[} {]} Experimentbeschreibungen vollständig - {[} {]}
Fehlerrechnung durchgeführt - {[} {]} Einheiten konsistent verwendet -
{[} {]} Theoretische Basis dokumentiert

\paragraph{Technische Wissenschaften
(Informatik/Elektronik)}\label{technische-wissenschaften-informatikelektronik}

\textbf{Dokumentationsanforderungen:} - Programmcode mit Versionierung -
Systemspezifikationen - Technische Zeichnungen und Schaltpläne - Test-
und Messprotokolle - Sicherheitsrichtlinien

\textbf{Qualitätskriterien:} - {[} {]} Code-Dokumentation vollständig -
{[} {]} Technische Zeichnungen normgerecht - {[} {]} Messverfahren
beschrieben - {[} {]} Sicherheitsaspekte berücksichtigt - {[} {]}
Versionierung implementiert

\paragraph{Ingenieurwesen
(KFZ-Technik)}\label{ingenieurwesen-kfz-technik}

\textbf{Dokumentationsanforderungen:} - Technische Spezifikationen -
Normen und Richtlinien - Mess- und Prüfprotokolle -
Wartungsdokumentation - Sicherheitsvorschriften

\textbf{Qualitätskriterien:} - {[} {]} Technische Daten vollständig -
{[} {]} Normverweise aktuell - {[} {]} Prüfprotokolle standardisiert -
{[} {]} Wartungsintervalle definiert - {[} {]} Sicherheitshinweise
dokumentiert

\subsubsection{Qualitätsdokumentation}\label{qualitaetsdokumentation}

\paragraph{Prüfprotokoll}\label{pruefprotokoll}

\begin{itemize}

\item
  Prüfdatum:
\item
  Fachbereich:
\item
  Angewandte Checkliste:
\item
  Prüfergebnis:
\item
  Anmerkungen:
\item
  Erforderliche Nacharbeiten:
\item
  Freigabevermerk:
\end{itemize}

\paragraph{Versionskontrolle}\label{versionskontrolle}

\begin{itemize}

\item
  Versionsnummer:
\item
  Änderungsdatum und Bearbeiter:
\item
  Änderungsbeschreibung:
\item
  Dokumentenstatus:
\item
  Freigabestatus:
\end{itemize}

\subsection{Vermeidung typischer
Fehler}\label{vermeidung-typischer-fehler}

\subsubsection{1. Klare Trennung von Originaltext und
Eigenleistung}\label{klare-trennung-von-originaltext-und-eigenleistung}

\textbf{Nicht korrekt:} \emph{„Die globale Erwärmung ist problematisch,
und meiner Meinung nach handeln Politiker zu langsam.<<}

\textbf{Korrekt:} \emph{„Die globale Erwärmung wird im Text als
problematische Entwicklung dargestellt. {[}Eigene Anmerkung: Die
politischen Reaktionszeiten erscheinen für effektive Gegenmaßnahmen zu
lang.{]}>>}

\subsubsection{2. Präzise
Quellenangaben}\label{praezise-quellenangaben}

\textbf{Nicht korrekt:} \emph{„Erneuerbare Energien bieten eine
nachhaltige Lösung.<<}

\textbf{Korrekt:} \emph{„Erneuerbare Energien werden als nachhaltige
Lösungsstrategie vorgeschlagen (vgl. Müller 2023, S. 15).<<}

%% Anhang
%\clearpage
%\appendix

\clearpage
\printbibliography
\end{document}
