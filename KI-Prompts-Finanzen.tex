% ju 28-Nov-24 KI-Prompts-Finanzen.tex
\documentclass{vorlage-design-main}
% Verwenden von fontspec und unicode-math für XeLaTeX oder LuaLaTeX

%% Ganze Überschrift
\title{}
%% Kürzerer Titel
\runningtitle{}
\author{Jan Unger}
\date{\today}

%% Referenzen
\addbibresource{literatur.bib}

\begin{document}

\maketitle

\begin{abstract}



\end{abstract}

\section{KI-Prompts}\label{ki-prompts}

\begin{itemize}

\item
  \hyperref[ki-prompts]{KI-Prompts}

  \begin{itemize}

  \item
    \hyperref[allgemeine-kommunikationsanweisung]{Allgemeine
    Kommunikationsanweisung}
  \item
    \hyperref[meine-notizen-pruxfcfen]{Meine Notizen prüfen}
  \item
    \hyperref[lernplan]{Lernplan}
  \item
    \hyperref[anweisung-zur-erstellung-einer-artikelzusammenfassung]{Anweisung
    zur Erstellung einer Artikelzusammenfassung}

    \begin{itemize}

    \item
      \hyperref[ankerwort-kernbotschaft]{Ankerwort: KERNBOTSCHAFT}
    \end{itemize}
  \item
    \hyperref[datenfluss-visualisieren-in-einem-strukturdiagramm]{Datenfluss
    visualisieren in einem Strukturdiagramm}
  \item
    \hyperref[finanzuxfcberblick]{Finanzüberblick}
  \item
    \hyperref[finanzuxfcbersicht]{Finanzübersicht}
  \item
    \hyperref[finanzanalyse]{Finanzanalyse}
  \item
    \hyperref[tabelle-tagebuch-um-tuxe4gliche-bargeldausgaben-zu-tracken]{Tabelle:
    Tagebuch, um tägliche Bargeldausgaben zu tracken}
  \end{itemize}
\end{itemize}

\begin{table}[ht]
  %\caption{}
  %\label{tab:my-table}
  \begin{tabular}{@{}ll@{}}
\toprule
Schritt
 &
Beschreibung
 \\
\midrule[\heavyrulewidth]
1 & \textbf{Erkläre} die wichtigsten Aspekte \\
2 & Prüfe und erstelle ein \textbf{Feedback} \\
3 & Erstelle eine \textbf{Markdown-Tabelle} \\
4 & Erstelle ein \textbf{Inhaltsverzeichnis}
\verb|von \#\# bis \#\#\#| in Markdown-Format mit
Ankerlinks \\
5 & Erstelle ein \textbf{Rechenbeispiel} \\
6 & \textbf{Ankerwort}: KERNBOTSCHAFT \\
7 & \textbf{Beschreibe} \\
8 & Fasse alle \textbf{wichtigen Daten, Fakten und Zahlen} zusammen \\
9 & Erstelle eine umfassende \textbf{Dokumentation} zum Script \\
\bottomrule
\end{tabular}%
\end{table}

\begin{itemize}

\item
  Welcher KI-Assistent und welchen Wissensstand hast du. Über welche
  Fähigkeiten verfügst du.

  \begin{itemize}

  \item
    Erstelle mir jeweils eine Anweisung
  \end{itemize}
\item
  \textbf{Analysiere} den folgenden Text auf sprachliche Besonderheiten.
\item
  Erkläre den \textbf{Unterschied zwischen}
\item
  Fasse die \textbf{Hauptargumente} in der Debatte um künstliche
  Intelligenz zusammen.
\item
  Entwerfe ein \textbf{Algorithmus} zur Berechnung von Primzahlen
\item
  Erstelle ein \textbf{Struktogramm} (Nassi-Shneiderman-Diagramm)
\item
  Entwickle einen einfachen \textbf{Python-Code} zur Berechnung von
  Primzahlen.
\item
  Entwickle einen einfachen \textbf{C++-Code} zur Berechnung von
  Primzahlen.
\end{itemize}

\subsection{Allgemeine
Kommunikationsanweisung}\label{allgemeine-kommunikationsanweisung}

\begin{itemize}

\item
  Einen sachlichen, neutralen und professionellen Kommunikationsstil
  verwenden, ohne direkte Anrede.
\item
  Präzise und unmissverständliche Formulierungen wählen.
\item
  \textbf{Fachliche Expertise} in spezifischen Themen wie Kfz-Technik,
  Softwareentwicklung, Elektronik, Mechanik, Finanzen.
\item
  Die deutsche Sprache für alle Kommunikationen nutzen.
\end{itemize}

\begin{enumerate}
\def\labelenumi{\arabic{enumi}.}

\item
  Erstelle eine \textbf{ToDo-Liste}
\item
  Erstelle eine \textbf{strukturierte Zusammenfassung} des Textes über
  >>\,<<

  \begin{itemize}

  \item
    Fasse die wesentlichen Konzepte und technischen Daten zusammen, die
    in den bereitgestellten Bildern beschrieben sind, einschließlich
    wichtiger Rechenbeispiele und relevanter Zahlen.
  \end{itemize}
\item
  Erstelle mir alle \textbf{Keywords} (bearbeiten)
\item
  \textbf{Erkläre} die Begriffe (unter Verwendung der Informationen aus
  den bereitgestellten Bildern/PDF)
\item
  Erstelle eine \textbf{Tabelle}
\item
  Erstelle in \textbf{Markdown mit LaTeX-Mathe} (verwende
  \verb|$...$ und $$...$$|)
\item
  Analysiere alle \textbf{Abbildungen und Tabellen} und Informationen
  aus grauen Kästen (unter Verwendung der Informationen aus den
  bereitgestellten Bildern/PDF)
\item
  Prüfe meine Notizen und erstelle ein \textbf{Feedback}
\item
  \textbf{Beschreibe} die Fehlersuche
\item
  Erstelle eine \textbf{Checkliste} für die Fehlersuche
\end{enumerate}

\subsection{Meine Notizen prüfen}\label{meine-notizen-pruefen}

Ich werde meine Notizen zur Verfügung stellen, prüfe und verbessere
unter folgenden Aspekten:

\begin{enumerate}
\def\labelenumi{\arabic{enumi}.}

\item
  Rechtschreibung und Grammatik
\item
  Inhaltliche Kohärenz und Logik
\item
  Stringenz und Eleganz des Ausdrucks
\item
  Argumentationsstruktur und -tiefe
\item
  Prägnanz und Klarheit der Formulierungen.
\item
  ohne direkte Anrede
\end{enumerate}

\begin{itemize}

\item
  Zielgruppe: Kfz-Meister und Bachelor-Studenten im Bereich
  Fahrzeugtechnik.
\item
  beachte: gebe gleichzeitig die wichtigsten Informationen (Keywords,
  Fachbegriffe, Fakten, Zahlen, Messwerte, Hinweise, Formeln,
  Zusammenhang) genau wieder.
\end{itemize}

\subsection{Lernplan}\label{lernplan}

\begin{itemize}

\item
  Gebe mir detailliertere Informationen zur Weiterbildung über das
  Thema.
\item
  Kannst du mir helfen, das fehlende Wissen zu erwerben
\item
  Erkläre mir, wie ich den Lernplan effektiv umsetzen kann.
\item
  Erstelle eine detaillierte Wöchentliche Lernplanstruktur basierend auf
  dem Lernplan.
\item
  Helfe mir die Lernplanstruktur an meine Bedürfnisse anzupassen. Der
  Personalisierte Plan berücksichtigt meine verfügbare Zeit, bevorzugten
  Lerntage und mein Ziel.
\end{itemize}

\subsection{Anweisung zur Erstellung einer
Artikelzusammenfassung}\label{anweisung-zur-erstellung-einer-artikelzusammenfassung}

\textbf{Einleitung:}

Beginnen Sie mit einem prägnanten einleitenden Satz, der das Hauptthema
des Artikels erfasst.

\textbf{Kernpunkte:}

\begin{enumerate}
\def\labelenumi{\arabic{enumi}.}

\item
  Fassen Sie die Hauptaussage des Artikels kurz und prägnant zusammen.
\item
  Listen Sie 5-7 Kernpunkte auf, die die wesentlichen Informationen und
  Argumente des Artikels wiedergeben. Achten Sie dabei auf:

  \begin{itemize}

  \item
    Kürze und Prägnanz (idealerweise ein Satz pro Punkt)
  \item
    Sachlichkeit und Neutralität
  \item
    Ausgewogenheit der dargestellten Perspektiven
  \item
    Chronologie oder logische Reihenfolge der Punkte
  \end{itemize}
\end{enumerate}

\textbf{Abschluss:} Formulieren Sie einen Abschlusssatz, der:

\begin{itemize}

\item
  Die übergreifende Bedeutung des Themas hervorhebt
\item
  Den Artikel in einen breiteren Kontext einordnet
\item
  Mögliche Implikationen oder Zukunftsaussichten andeutet, ohne
  spekulativ zu werden
\item
  Die Neutralität wahrt und keine persönliche Meinung ausdrückt
\end{itemize}

\textbf{Überprüfung:}

\begin{itemize}

\item
  Überprüfen Sie die Gesamtlänge der Zusammenfassung. Sie sollte
  kompakt, aber informativ sein und idealerweise nicht mehr als 200-250
  Wörter umfassen.
\item
  Stellen Sie sicher, dass die Zusammenfassung für sich allein
  verständlich ist, ohne dass der Leser den Originalartikel kennen muss.
\end{itemize}

\begin{center}\rule{0.5\linewidth}{0.5pt}\end{center}

\subsubsection{Ankerwort: KERNBOTSCHAFT}\label{ankerwort-kernbotschaft}

Dieses Ankerwort kann als Gedächtnisstütze für die Erstellung einer
effektiven Zusammenfassung dienen:

\begin{itemize}

\item
  \textbf{K} - Kurzer Einleitungssatz
\item
  \textbf{E} - Essentielle Punkte auflisten
\item
  \textbf{R} - Relevanz hervorheben
\item
  \textbf{N} - Neutrale Darstellung
\item
  \textbf{B} - Breiteren Kontext einbeziehen
\item
  \textbf{O} - Ohne Spekulation
\item
  \textbf{T} - Eigenständig verständlich
\item
  \textbf{S} - Sachlich und prägnant
\item
  \textbf{C} - Checkup der Gesamtlänge
\item
  \textbf{H} - Hauptaussagen erfassen
\item
  \textbf{A} - Abschlussformulierung
\item
  \textbf{F} - Finale Überprüfung
\item
  \textbf{T} - Thematische Einordnung
\end{itemize}

\subsection{Datenfluss visualisieren in einem
Strukturdiagramm}\label{datenfluss-visualisieren-in-einem-strukturdiagramm}

\begin{itemize}
\item
  Anweisung zur erstellung eines Diagramms, das die Gesamtstruktur und
  den Datenfluss visualisiert. Dieses Diagramm wird als Mermaid-Graph
  implementiert, um eine klare und übersichtliche Darstellung zu
  gewährleisten
\item
  Erstelle eine \textbf{README-Datei zur Dokumentation} des Codes sowie
  ein \textbf{Strukturdiagramm zur Visualisierung} des Datenflusses
  (Zusammenhang).

  \begin{itemize}

  \item
    README: Projektübersicht, Installationsanleitung für jede Plattform
    (arduino/, platformio/, esp-idf/), Verwendungsanleitung,
    Fehlerbehebung, Beitragsrichtlinien, MIT License (Copyright (c)
    {[}2024{]} {[}Jan Unger{]})
  \item
    Die Farbkombination sollen eine gute Balance zwischen visueller
    Unterscheidung der verschiedenen Programmelemente und Lesbarkeit
    bieten. Die weißen Beschriftungen auf den farbigen Hintergründen
    sorgen für einen starken Kontrast und gute Lesbarkeit.
  \end{itemize}
\item
  Erstelle ein \textbf{Struktogramm} (Nassi-Shneiderman-Diagramm)

  \begin{itemize}

  \item
    Die Farbkombination sollen eine gute Balance zwischen visueller
    Unterscheidung der verschiedenen Programmelemente und Lesbarkeit
    bieten. Die weißen Beschriftungen auf den farbigen Hintergründen
    sorgen für einen starken Kontrast und gute Lesbarkeit.
  \item
    \textbf{Was soll das Programm machen?} Beschreibung: Hallo Welt!
    ausgeben und die Programmierumgebung testen.
  \end{itemize}
\item
  Erstelle jeweils ein \textbf{Projekt}: main (Hallo Welt) \textbf{für
  verschiedene Entwicklungsumgebungen} Arduino IDE und in VSCode mit
  PlatformIO und EspIDF (ESP32-Dev-Board), Verwende eine effiziente
  Projektverwaltung für ein solches Multi-Plattform-Embedded-Projekt.
\end{itemize}

\subsection{Finanzüberblick}\label{finanzueberblick}

\textbf{Daten bereinigen}: Diba-Filter >>Drei Monate<< als .csv
exportieren

\textbf{Berechne alle Einnahmen und Ausgaben} für die \textbf{Woche}:
und erstelle eine Tabelle mit Ausgabenkategorisierung.

Allgemeine Anweisungen

\begin{itemize}

\item
  Erstelle einen Finanzüberblick je \textbf{Monat} Mai, Juni, Juli
  basierend auf den Umsätzen des Girokontos. Dabei berücksichtige die
  vorgegebene 01-Finanzueberblick-Vorlage.
\item
  Fixkosten und variablen Kosten genauer analysieren
\item
  sonstigen Ausgaben genauer analysieren
\item
  Möglichkeiten zur Kostenoptimierung aufzeigen
\end{itemize}

\subsection{Finanzübersicht}\label{finanzuebersicht}

\textbf{Prüfe und erstelle ein Feedback} je \textbf{Monat} Mai, Juni,
Juli

\begin{itemize}

\item
  Erstelle eine kombinierte Darstellung aus Tabelle und detaillierter
  Liste für meine Finanzübersicht
\item
  Einnahmen und Ausgaben (Kategorie)
\item
  Detaillierte Ausgabenkategorisierung
\end{itemize}

\subsection{Finanzanalyse}\label{finanzanalyse}

\textbf{Prüfe und erstelle ein Feedback} je \textbf{Monat} Mai, Juni,
Juli

\textbf{Notfallfonds und Schuldentilgung:}

\begin{itemize}

\item
  Nutze den monatlichen Überschuss wie folgt:

  \begin{itemize}

  \item
    50\% für den Aufbau eines Notfallfonds
  \item
    50\% für zusätzliche Tilgungen des Annuitätendarlehens
  \end{itemize}
\end{itemize}

\textbf{Monatliches Ausgaben-Tracking}:

\begin{itemize}

\item
  Erstelle eine detaillierte Übersicht meiner monatlichen Ausgaben.
  Vergleiche diese mit dem Durchschnitt der letzten drei Monate:
\item
  Führe ein detailliertes Tagebuch für meine Bargeldausgaben.
\item
  Überprüfe regelmäßig meine Abonnements und Online-Dienste auf
  Notwendigkeit und mögliche günstigere Alternativen.
\item
  Versuche, die Kategorie >>Sonstiges<< weiter aufzuschlüsseln, um ein
  klareres Bild von diesen Ausgaben zu erhalten.
\item
  Setze realistische Sparziele für die größten Ausgabenkategorien, z.B.
  eine 5\%ige Reduzierung der Lebensmittelausgaben im nächsten Monat.
\end{itemize}

\textbf{Ausgabenoptimierung}:

\begin{itemize}

\item
  Identifiziere die Top 3 variablen Ausgaben und setze realistische
  Reduktionsziele
\end{itemize}

\textbf{Regelmäßige Finanzüberprüfung}: (Führe vierteljährliche
Finanz-Check-ups durch)

\begin{itemize}

\item
  Vergleiche Ist- mit Soll-Zustand
\item
  Passe Ziele und Strategien entsprechend an
\end{itemize}

\subsection{Tabelle: Tagebuch, um tägliche Bargeldausgaben zu
tracken}\label{tabelle-tagebuch-um-taegliche-bargeldausgaben-zu-tracken}

\begin{table}[ht]
  %\caption{}
  %\label{tab:my-table}
  \begin{tabular}{@{}llll@{}}
  \toprule

Datum & Betrag (EUR) & Kategorie & Beschreibung/Zweck \\
\midrule[\heavyrulewidth]
01 & & & \\
02 & & & \\
03 & & & \\
04 & & & \\
05 & & & \\
06 & & & \\
07 & & & \\
Wochensumme: & & & \\
  \bottomrule
  \end{tabular}%
\end{table}

%% Anhang
%\clearpage
%\appendix

\clearpage
\printbibliography
\end{document}
