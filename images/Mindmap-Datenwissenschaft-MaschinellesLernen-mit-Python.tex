\documentclass[tikz, border=10pt]{standalone}
\usepackage[T1]{fontenc} % Verwendung von Vektorschriften
\usepackage[ngerman]{babel} % Deutsche Rechtschreibung und Silbentrennung
\usepackage{tikz}
\usetikzlibrary{mindmap, shapes.geometric} % Import der shapes.geometric Bibliothek

% Farbdefinitionen
\definecolor{MindMapBlue}{RGB}{0, 105, 180}
\definecolor{MindMapGreen}{RGB}{76, 153, 0}
\definecolor{MindMapRed}{RGB}{237, 28, 36}
\definecolor{MindMapYellow}{RGB}{255, 242, 0}
\definecolor{MindMapOrange}{RGB}{255, 127, 39}
\definecolor{MindMapPurple}{RGB}{150, 111, 214}
\definecolor{MindMapGray}{RGB}{195, 195, 195}
\definecolor{MindMapWhite}{RGB}{255, 255, 255}
\definecolor{MindMapBlack}{RGB}{0, 0, 0}

\begin{document}
\begin{tikzpicture}[mindmap, grow cyclic,
                    every node/.style={concept, fill=none, text width=2.5cm, font=\small, align=center, line width=0.5pt},
                    level 1/.append style={sibling angle=360/5},
                    level 2/.append style={level distance=4cm, sibling angle=45},
                    level 3/.style={sibling angle=45}, % Unterunterknoten
                    level 4/.style={sibling angle=45},
                    edge from parent/.style={concept color=MindMapBlack, line width=0.5pt},
                    text only/.style={draw=none, rectangle, text=MindMapBlack}]

% Hauptknoten
\node[text width=4cm, font=\large\bfseries]{Datenwissenschaft und Maschinelles Lernen mit Python}
    child [concept color=MindMapBlue] { node {Grundlegende Bibliotheken}
        child { node {Pandas}
            child { node [text only] {Datenmanipulation}}
            child { node [text only] {Datenreinigung}}
            child { node [text only] {Zeitreihenanalyse}}
        }
        child { node {NumPy}
            child { node [text only] {Mehrdimensionale Arrays}}
            child { node [text only] {Mathematische Operationen}}
        }
        child { node {SciPy}
            child { node [text only] {Wissenschaftliches Rechnen}}
            child { node [text only] {Statistik}}
            child { node [text only] {Optimierungs-aufgaben}}
        }
    }
    child [concept color=MindMapGreen] { node {Daten-visualisierung}
        child { node {Matplotlib}
            child { node [text only] {Basisdiagramme und -grafiken}}
            child { node [text only] {Anpassung von Plots}}
        }
        child { node {Seaborn}
            child { node [text only] {Statistische Grafiken}}
            child { node [text only] {Attraktive Standarddesigns}}
        }
        child { node {Plotly}
            child { node [text only] {Interaktive, webbasierte Grafiken}}
            child { node [text only] {Komplexere Visualisierungen}}
        }
    }
    child [concept color=MindMapRed] { node {Maschinelles Lernen}
        child { node {Traditionelles Maschinelles Lernen}
            child { node {Scikit-learn}
                child { node [text only] {Supervised Learning}}
                child { node [text only] {Unsupervised Learning}}
            }
        }
        child { node {Tiefes Lernen}
            child { node {TensorFlow}
                child { node [text only] {Aufbau tiefer neuronaler Netzwerke}}
                child { node [text only] {Anwendungs-bereiche}}
            }
            child { node {PyTorch}
                child { node [text only] {Dynamische Berechnungsgraphen}}
                child { node [text only] {Forschungsfreundlich}}
            }
        }
    }
    child [concept color=MindMapOrange] { node {Entwicklungs-umgebung}
        child { node {Jupyter Notebooks}
            child { node [text only] {Interaktive Code-Ausführung}}
            child { node [text only] {Integration von Text, Code und Visualisierungen}}
        }
    }
    child [concept color=MindMapPurple] { node {Schlüsselkonzepte}
        child { node [text only] {Datenanalyse}}
        child { node [text only] {Statistische Grafiken}}
        child { node [text only] {Explorative Datenanalyse}}
        child { node [text only] {Algorithmen}}
        child { node [text only] {Regression und Klassifikation}}
    };
\end{tikzpicture}
\end{document}
