% ju 26-Nov-24 Python-Paket-Management.tex
\documentclass{vorlage-design-main}

% Checkbox Definition
\newcommand{\checkbox}{\square}


%% Ganze Überschrift
\title{}
%% Kürzerer Titel zur Verwendung im Seitenkopf
\runningtitle{}
\author{Jan Unger}
\date{\today}

%% Die .bib-Datei mit vollständigen Referenzen zur Verwendung mit biblatex. 
%% articleclass lädt das Paket biblatex-chicago mit Anpassungen
\addbibresource{literatur.bib}

\begin{document}

\maketitle

\begin{abstract}



\end{abstract}

\section{Python Paket-Management unter
macOS}\label{python-paket-management-unter-macos}

\subsection{Virtuelle Umgebung
einrichten}\label{virtuelle-umgebung-einrichten}

\begin{lstlisting}[language=bash]
# Terminal öffnen und zum Projektverzeichnis navigieren
# Virtuelle Umgebung deaktivieren
deactivate
# Optional: Virtuelle Umgebung löschen
rm -rf venv
# Virtuelle Umgebung erstellen
python3 -m venv venv
# Virtuelle Umgebung aktivieren
source venv/bin/activate
\end{lstlisting}

\subsection{Pakete installieren}\label{pakete-installieren}

\begin{table}[ht]
  %\caption{}
  %\label{tab:my-table}
  \begin{tabular}{@{}ll@{}}
\toprule
\textbf{Paket}
 & 
\textbf{Funktionalität}
 \\
\midrule[\heavyrulewidth]
\verb|PyPDF2| & Verarbeitung von PDF-Dateien (z. B.
Extraktion von Inhalten). \\
\verb|pytube| & Herunterladen von Videos von
YouTube. \\
\verb|whisper| & Sprach-zu-Text-Transkription (z. B.
OpenAI Whisper). \\
\verb|youtube-transcript-api| & Abrufen von
Transkripten von YouTube-Videos. \\
\verb|black|, \verb|isort|,
\verb|flake8| & Tools für Code-Formatierung,
Import-Sortierung und Linting. \\
\verb|mypy| & Statische Typprüfung für
Python-Code. \\
\verb|Pillow| & Bildbearbeitung (z. B. Extraktion und
Verarbeitung von Bildern). \\
\verb|PyMuPDF| & Verarbeitung von PDFs (Extraktion
von Bildern, Texten etc.). \\
\verb|rich| & Formatierte Konsolenausgabe (z. B.
Fortschrittsbalken, farbige Logs). \\
\verb|Jinja2| & Template-Engine für Python (z. B.
HTML-Generierung). \\
\verb|PyYAML| & Verarbeitung von YAML-Dateien
(Konfiguration, Datenaustausch). \\
\verb|tqdm| & Fortschrittsanzeigen für Schleifen und
Iterationen. \\
\verb|requests| & HTTP-Bibliothek für API-Anfragen
und Webzugriffe. \\
\verb|certifi| & Zertifikate für sichere
HTTPS-Verbindungen. \\
\verb|typing\_extensions| & Erweiterte Typhinweise
für Python \textless{} 3.10. \\
\bottomrule
\end{tabular}%
\end{table}

Die Pakete bilden ein kohärentes Set:

\begin{itemize}

\item
  Medienverarbeitung (Video, Audio, PDF, Bilder)
\item
  Entwicklungswerkzeuge (Formatierung, Linting, Typing)
\item
  Webinteraktion (Downloads, API-Zugriffe)
\item
  Ausgabeformatierung und Benutzerinteraktion
\end{itemize}

\begin{lstlisting}[language=bash]
# pip aktualisieren
pip install --upgrade pip
# Benötigte Pakete installieren
pip install PyPDF2 pytube whisper youtube-transcript-api PyMuPDF Pillow rich
# HTML & PDF
pip install Jinja2 MarkupSafe PyYAML tqdm 
# Tools installieren
pip install black isort flake8 mypy pipdeptree
\end{lstlisting}

\subsection{Pakete in requirements.txt
speichern}\label{pakete-in-requirements.txt-speichern}

\begin{lstlisting}[language=bash]
# Alle installierten Pakete in requirements.txt speichern
pip freeze > requirements.txt
# Inhalt anzeigen
cat requirements.txt
# Alle Pakete aus der requirements.txt installieren
pip install -r requirements.txt
\end{lstlisting}

\subsection{Pakete aktualisieren}\label{pakete-aktualisieren}

\begin{lstlisting}[language=bash]
# Aktuelle Paketliste anzeigen
pip list
# Veraltete Pakete anzeigen
pip list --outdated
# Alle Pakete aktualisieren
pip list --outdated | cut -d ' ' -f1 | tail -n +3 | xargs -n1 pip install -U
# Nach der Aktualisierung requirements.txt erneuern
pip freeze > requirements.txt
\end{lstlisting}

\subsection{Troubleshooting}\label{troubleshooting}

\begin{lstlisting}[language=bash]
# Cache leeren bei Problemen
pip cache purge

# Paket neu installieren
pip uninstall PyPDF2
pip install PyPDF2

# Konfliktprüfung
pip check

# Requirements als Graph visualisieren
pip install pipdeptree
pipdeptree
\end{lstlisting}

\subsection{Code-Qualität prüfen}\label{code-qualitaet-pruefen}

\begin{lstlisting}[language=bash]
pip install black isort flake8 mypy
# black: Gibt aus, welche Dateien neu formatiert wurden.
black create_gallery.py extract_pdf_images.py pdf_extractor.py
# isort: Ändert die Importreihenfolge oder zeigt fehlerhafte Sortierungen an.
isort create_gallery.py extract_pdf_images.py pdf_extractor.py
# flake8: Gibt Warnungen oder Fehler basierend auf PEP8-Konventionen aus
flake8 create_gallery.py extract_pdf_images.py pdf_extractor.py
# mypy: Gibt Typfehler aus, falls die Typannotationen nicht korrekt sind.
mypy create_gallery.py extract_pdf_images.py pdf_extractor.py

# PDF Kapitel Extraktor *.pdf, Seiten von bis?
python pdf_extractor.py
# Extrahiere Bilder aus *.pdf
python extract_pdf_images.py
# Gallerie: gallery/*_gallery.html
python create_gallery.py

# aktuelle Verzeichnis ($(pwd)) wird zum Python-Suchpfad hinzugefügt. Python Module und Pakete, die sich in diesem Verzeichnis befinden, erkennt und importieren kann.
export PYTHONPATH="${PYTHONPATH}:$(pwd)"
# Im Entwicklungsmodus wird das Paket nicht kopiert und installiert, sondern als Referenz zur aktuellen Verzeichnisstruktur verlinkt.
# Änderungen am Quellcode sind sofort in der Umgebung verfügbar, ohne dass man das Paket erneut installieren muss.
# Voraussetzung: setup.py
pip install -e .
\end{lstlisting}

\textbf{Konfiguration:}

\begin{lstlisting}[language=bash]
# .flake8
[flake8]
max-line-length = 100
exclude = venv, .git

# pyproject.toml (für black und isort)
[tool.black]
line-length = 100

[tool.isort]
profile = "black"
line_length = 100
\end{lstlisting}

\subsection{Projektspezifische
requirements.txt}\label{projektspezifische-requirements.txt}

\begin{lstlisting}
# requirements.txt
black==24.10.0
certifi==2024.8.30
charset-normalizer==3.4.0
click==8.1.7
defusedxml==0.7.1
flake8==7.1.1
idna==3.10
isort==5.13.2
markdown-it-py==3.0.0
mccabe==0.7.0
mdurl==0.1.2
mypy==1.13.0
mypy-extensions==1.0.0
packaging==24.2
pathspec==0.12.1
pillow==11.0.0
pipdeptree==2.23.4
platformdirs==4.3.6
pycodestyle==2.12.1
pyflakes==3.2.0
Pygments==2.18.0
PyMuPDF==1.24.14
PyPDF2==3.0.1
pytube==15.0.0
requests==2.32.3
rich==13.9.4
six==1.16.0
typing_extensions==4.12.2
urllib3==2.2.3
whisper==1.1.10
youtube-transcript-api==0.6.3
\end{lstlisting}

\section{Projektstruktur}\label{projektstruktur}

\begin{lstlisting}
# Projektstruktur anzeigen
tree -L 2 --dirsfirst
# Projektstruktur erfolgt nach Python-Projektstandards
# Ausgabe kommentieren
\end{lstlisting}

\subsection{Tipps und Best Practices}\label{tipps-und-best-practices}

\begin{enumerate}
\def\labelenumi{\arabic{enumi}.}

\item
  \textbf{Virtuelle Umgebung}:

  \begin{itemize}
  
  \item
    Immer in einer virtuellen Umgebung arbeiten
  \item
    Für jedes Projekt eine separate Umgebung erstellen
  \item
    Namen der virtuellen Umgebung (venv) in .gitignore aufnehmen
  \end{itemize}
\item
  \textbf{Requirements}:

  \begin{itemize}
  
  \item
    Requirements.txt regelmäßig aktualisieren
  \item
    Version in Kommentaren dokumentieren
  \item
    Backup der funktionierenden requirements.txt anlegen
  \end{itemize}
\item
  \textbf{Sicherheit}:

  \begin{itemize}
  
  \item
    Regelmäßig nach Sicherheitsupdates suchen
  \item
    Pakete nur aus vertrauenswürdigen Quellen installieren
  \item
    Bei Produktivnutzung fixe Versionen verwenden
  \end{itemize}
\end{enumerate}

\subsection{Automatisierungsskript I}\label{automatisierungsskript-i}

\begin{lstlisting}[language=bash]
#!/bin/bash
# update_python_packages.sh

echo "Starte Paket-Update..."

# Aktiviere virtuelle Umgebung
source venv/bin/activate

# Backup von requirements.txt
cp requirements.txt requirements.backup.txt

# Aktualisiere pip
pip install --upgrade pip

# Aktualisiere alle Pakete
pip list --outdated | cut -d ' ' -f1 | tail -n +3 | xargs -n1 pip install -U

# Erstelle neue requirements.txt
pip freeze > requirements.txt

echo "Update abgeschlossen. Neue Paketversionen:"
cat requirements.txt

# Deaktiviere virtuelle Umgebung
#deactivate
\end{lstlisting}

Nutzung:

\begin{lstlisting}[language=bash]
chmod +x update_python_packages.sh
#source venv/bin/activate
./update_python_packages.sh
\end{lstlisting}

\subsection{Automatisierungsskript I}\label{automatisierungsskript-i-1}

\begin{lstlisting}[language=bash]
#!/bin/bash
# check_pythoncode_quality.sh

# Skript zur Code-Qualitätsprüfung
echo "Starte Code-Qualitätsprüfung..."

# Definiere Verzeichnis für Python-Skripte
SCRIPT_DIR="python-scripte"

# Definiere die zu prüfenden Python-Dateien
PYTHON_FILES="$SCRIPT_DIR/dateien_inhaltsverzeichnis.py \
              $SCRIPT_DIR/dokumentation.py \
              $SCRIPT_DIR/git_hilfsprogramm.py \
              $SCRIPT_DIR/html1_konverter_pandoc.py \
              $SCRIPT_DIR/html2_dateien_verarbeiten.py \
              $SCRIPT_DIR/html3_navigation.py \
              $SCRIPT_DIR/html4_entfernen.py \
              $SCRIPT_DIR/latex_convert1.py \
              $SCRIPT_DIR/latexcode_entfernen2.py \
              $SCRIPT_DIR/suchen_ersetzen.py \
              $SCRIPT_DIR/sync_tex.py \
              image_resizer.py \
              scriptauswahl.py"

# Prüfe, ob das Verzeichnis existiert
if [ ! -d "$SCRIPT_DIR" ]; then
    echo "Fehler: Verzeichnis $SCRIPT_DIR nicht gefunden!"
    exit 1
fi

# Führe die Prüfungen aus
echo "=== Führe Black aus ==="
black $PYTHON_FILES

echo -e "\n=== Führe isort aus ==="
isort $PYTHON_FILES

echo -e "\n=== Führe flake8 aus ==="
flake8 $PYTHON_FILES

echo -e "\n=== Führe mypy aus ==="
mypy $PYTHON_FILES

echo -e "\nCode-Qualitätsprüfung abgeschlossen."
\end{lstlisting}

Nutzung:

\begin{lstlisting}[language=bash]
chmod +x check_pythoncode_quality.sh
#source venv/bin/activate
./check_pythoncode_quality.sh
\end{lstlisting}


%% Anhang
%\clearpage
%\appendix

\clearpage
\printbibliography
\end{document}